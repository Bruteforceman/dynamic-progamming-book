\newcommand{\mc}[2]{\multicolumn{#1}{c}{#2}}
\definecolor{Gray}{gray}{0.85}
\definecolor{LightCyan}{rgb}{0.88,1,1}

\newcolumntype{g}{>{\columncolor{Gray}}c}

\chapter{ডিজিট ডিপি}

কিছু কিছু সমস্যায় তোমাকে কোন একটা রেঞ্জের মধ্যে বিশেষ কোন ধর্ম সিদ্ধ করে এমন পূর্নসংখ্যা নিয়ে কাজ করতে হয়। এমন সমস্যা দেখলে মনে হয় হয়ত গাণিতিক কোন ধর্ম ব্যবহার করে এগুলো সমাধান করতে হবে। এই ধরনের সমস্যাও যে ডাইনামিক প্রোগ্রামিং দিয়ে সমাধান করা যায় তা সহজে আন্দাজ করা যায় না। ডিজিট ডিপি এমনই একটি টেকনিক। আমরা এ পর্যন্ত যেসব প্রবলেম দেখেছি তার চেয়ে এটি বেশ ভিন্ন ধরনের। তবে মুল আইডিয়াটা ধরতে পারলে এটি মোটেও কঠিন কোন ডিপি নয়। 

\section{সংখ্যা নিয়ে কিছু কথা}
ডিজিট ডিপি বুঝতে হলে আমরা দুটি পূর্নসংখ্যা কীভাবে তুলনা করি সেটা ভালোভাবে বুঝতে হবে। দুটি সংখ্যা দেওয়া থাকলে কোনটি কোনটি ছোট সেটা হয়ত একটা বাচ্চাও বলতে পারবে। কিন্তু আমরা সংখ্যা তুলনা করার সময় যে অ্যালগরিদম ব্যবহার করলাম (মনের অজান্তে হলেও) সেটা নিয়ে চিন্তা করি না। ডিজিট ডিপি বোঝার জন্য আমাদের এই প্রসেসটার একটু গভীরে যেতে হবে। একটি উদাহরণ দেখা যাক। ধর তোমার কাছে দুটি সংখ্যা $a = 56744$ এবং $b = 56729$ দেওয়া আছে। তোমাকে বলতে হবে কোনটা বড়। এর জন্য আমরা যেটা করি তা হল সংখ্যা দুটির অঙ্কগুলোকে বাম থেকে ডান দিকে এক এক করে তুলনা করতে থাকি। প্রথম যে সংখ্যাতে ছোট ডিজিট পাবো সেটাকেই ছোট সংখ্যা বলে ঘোষণা করে দিতে পারব। নিচের ছবিটা দেখ। $a$ আর $b$ এর ডিজিটগুলোকে নিচে নিচে লেখেছি।

\newpage

\begin{center}
\begin{tabular}{ |c|c|c|g|c|c| }
 \hline
 5 & 6 & 7 & 4 & 4 \\
 \hline
\end{tabular} \\ 
\vspace{5mm}
\begin{tabular}{ |c|c|c|g|c|c| }
 \hline
 5 & 6 & 7 & 2 & 9 \\
 \hline
\end{tabular}
\end{center}

আমরা বাম দিকে থেকে ডিজিটগুলো এক এক করে তুলনা করেছি এবং চতুর্থ ডিজিটে প্রথম ভিন্ন ভিন্ন অঙ্ক পেয়েছি (mismatch পেয়েছি)। উপরের সংখ্যার অঙ্কটি বড় তাই উপরেরটিই বড় সংখ্যা। একটা জিনিশ খেয়াল কর। চতুর্থ ডিজিটের পর কোন কোন ডিজিট আসলো তা কিন্তু আমাদের আর দেখারই দরকার নাই। প্রথম যে পজিশনে ভিন্ন ভিন্ন অঙ্ক পাওয়া গেছে সেটা দিয়েই সংখ্যা দুটি তুলনা করা যাবে। এখানে $a$ আর $b$ তে একই সংখ্যক অঙ্ক ছিল বলে আমাদের সুবিধা হয়েছে। কিন্তু দুটিতে একই সংখ্যক অঙ্ক না থাকলেও কিন্তু আমরা আগে কিছু শূন্য বসিয়ে দুটিকে সমান ডিজিট বিশিষ্ট সংখ্যা বানিয়ে নিতে পারতাম। তাই এই অ্যালগরিদম আসলে যেকোনো দুটি সংখ্যা তুলনা করার ক্ষেত্রেই খাটবে। আর এই আইডিয়া ব্যবহার করেই ডিজিট ডিপির সব কাজ করা হয়। 

এবার একটু ভিন্ন দিকে আসা যাক। ধর তোমাকে $123456$ এর চেয়ে ছোট একটা সংখ্যা বানাতে বলা হল। কিন্তু তোমার ছোট ভাই এসে বাম দিকের কিছু অঙ্ক অলরেডি বসিয়ে দিয়েছে। তোমাকে বাকি অঙ্কগুলো পূরণ করতে হবে। যেমন নিচের সংখ্যাতে তোমার ভাই প্রথম তিনটা সংখ্যা বসিয়ে দিয়েছে 

\begin{center}
\begin{tabular}{ |c|c|c|c|c|c| }
 \hline
 1 & 2 & 0 & & & \\
 \hline
\end{tabular} \\ 
\end{center}

এখানে তুমি বাকি দুটি ঘরে যে অঙ্কই বসাও না কেন সংখ্যাটি $123456$ এর চেয়ে ছোট হবে।  কারণ $123456$ এর তৃতীয় ডিজিট $3$ কিন্তু আমাদের তৈরি করা সংখ্যাতে তৃতীয় ডিজিট $0$। তাই বাকি ঘরগুলোতে যেটাই বসাও না কেন $123456$ এর চেয়ে বড় সংখ্যা পাওয়া সম্ভব নয়। 

কিন্তু যদি তোমার ছোট ভাইয়ের বসানো সংখ্যাগুলো এমন হয় 

\begin{center}
\begin{tabular}{ |c|c|c|c|c|c| }
 \hline
 1 & 2 & 5 & & & \\
 \hline
\end{tabular} \\ 
\end{center}

তাহলে তুমি বাকি ঘরগুলোতে যাই বসাও না কেন $123456$ এর চেয়ে ছোট সংখ্যা বানাতে পারবে না (একই কারণ)। আরেকটা কেইস আছে। সেটা হল যদি বসানো সংখ্যাগুলো এমন হয় 

\begin{center}
\begin{tabular}{ |c|c|c|c|c|c| }
 \hline
 1 & 2 & 3 & ? & & \\
 \hline
\end{tabular} \\ 
\end{center}
 
 এ ক্ষেত্রে তোমার কিছু বাধ্যবাধকতা আছে। ? চিহ্নিত ঘরটাতে তুমি যেকোনো সংখ্যা বসাতে পারবে না। তোমাকে সেখানে অবশ্যই $4$ এর সমান বা ছোট একটি ডিজিট বসাতে হবে, নাহলে সংখ্যাটি বড় হয়ে যাবে। 
 
আমাদের আলোচনার মূল পয়েন্ট হল তুমি যদি বাম থেকে ডান দিকে ডিজিট বসাতে থাক তাহলে কোন পজিশনে ডিজিট বসানোর সময় কেবল এটা জানাই যথেষ্ট যে মূল সংখ্যার ডিজিটগুলোর সাথে আমাদের বানানো সংখ্যার ডিজিটগুলোর কোথাও মিসম্যাচ (mismatch) হয়েছে কিনা, অর্থাৎ মূল সংখ্যা থেকে ভিন্ন কোনো ডিজিট কোনো পজিশনে বসিয়েছি কিনা। যদি বসিয়ে থাকি তাহলে পরবর্তী ফাঁকা ঘরটাতে আমরা যেকোনো ডিজিট বসাতে পারব। আর যদি না বসিয়ে থাকি তাহলে ফাঁকা ঘরটিতে এমন একটি ডিজিট বসাতে হবে যেন তা মূল সংখ্যার ডিজিটের চেয়ে বড় না হয়ে যায়।
