\chapter{এক্সচেঞ্জ আর্গুমেন্ট}

\section{প্রমাণ দাও}

সাধারণত গ্রিডি অ্যালগরিদম গুলো অনেকটা এরকম হয়ঃ যতক্ষণ পর্যন্ত সম্ভব প্রদত্ত শর্তগুলো ঠিক রেখে তুমি প্রতিবার একটি করে ইলিমেন্ট সিলেক্ট করে তোমার সলিউশনে অ্যাড করবা যেটায় তোমার সবচেয়ে বেশি লাভ হয়। আমরা এক্সচেঞ্জ আর্গুমেন্ট ব্যবহার করে যেমন আমাদের এই গ্রিডি অ্যালগরিদমের শুদ্ধতা প্রমাণ করতে পারি, তেমনি এক্সচেঞ্জ আর্গুমেন্ট এর ধাপ গুলো নিয়ে চিন্তা করতে গিয়ে আমাদের গ্রিডি সলিউশনও দাঁড় করিয়ে ফেলতে পারি। এক্সচেঞ্জ আর্গুমেন্ট প্রুফ গুলোর মেইন আইডিয়া হলো, তুমি যেকোনো একটি অপ্টিমাল সলিউশন নিবে, তারপর সেটিকে ধাপে ধাপে এমনভাবে তোমার গ্রিডি সলিউশনে পরিবর্তন করবে যেন প্রতি ধাপে তোমার কোন লস না হয়। তাহলে তুমি বলতে পারবে তোমার গ্রিডি সলিউশন অন্তত কোন একটি অপ্টিমাল সলিউশনের চাইতে খারাপ না। অন্যভাবে বলতে গেলে, তোমার সলিউশনও একটি অপ্টিমাল সলিউশন। একটা উদাহরণ দেখা যাক।

\begin{problem}[ডট প্রডাক্ট মিনিমাইজেশন]
তোমাকে দুটি অ্যারে দেওয়া আছে। তোমাকে এমনভাবে অ্যারে দুটিকে রিঅ্যারেঞ্জ করতে হবে যেন তাদের ডট গুণফল অর্থাৎ, $\sum_{i=1}^{N} A_i B_i$ এর মান মিনিমাম হয়।
\end{problem}
\begin{solution}
আমরা চাই না দুটি বড় বড় সংখ্যা একসাথে থাকুক কারণ তাদের গুণফল অবশ্যই বড় হয়ে যাবে। অন্যদিকে, দুটি ছোট ছোট সংখ্যা একসাথে থাকলে লাভ হতে পারে বলে মনে হতে পারে। কিন্তু এরকম করলে বড় বড় সংখ্যা গুলো একসাথে হয়ে যাবে। তাহলে এরকম একটা কিছু করা যায়- একটি ছোট আর একটি বড় সংখ্যা একসাথে পেয়ারআপ করা। এই আইডিয়াটাকে গুছিয়ে বললে হবে- প্রথম অ্যারেটিকে নন-ডিক্রিজিং অর্ডারে সর্ট করা এবং দ্বিতীয় অ্যারেটিকে নন-ইনক্রিজিং অর্ডারে সর্ট করা। এখন আমাদের প্রমাণ করতে হবে, এটি একটি অপ্টিমাল সলিউশন। আমরা ধরে নিতে পারি প্রথম অ্যারেটি নন-ডিক্রিজিং অর্ডারে সর্ট করা আছে। এখন ধরো এমন একটা অপ্টিমাল সলিউশন আছে যেখানে $B$ ডিক্রিজিং অর্ডারে সর্ট করা নেই, অর্থাৎ, এমন একটা $i$ আছে যেন, $B_{i} < B_{i+1}$।  এখন আমরা এদেরকে সোয়াপ করে আমাদের গ্রিডি সলিউশনের দিকে যেতে চাই। যদি সোয়াপ করি, তাহলে আমদের গুণফলে যেই অতিরিক্ত কস্ট অ্যাড হবে তা হলোঃ $A_iB_{i+1} + A_{i+1}B_i - A_iB_i - A_{i+1}B_{i+1}$।  সুতরাং আমাদের প্রমাণ করতে হবে-
\begin{align*}
	A_iB_{i+1} + A_{i+1}B_i - A_iB_i - A_{i+1}B_{i+1} &\le 0 & \\
	A_i(B_{i+1} - B_i) - A_{i+1}(B_{i+1} - B_i) &\le 0\\
	A_i &\le A_{i+1} &\text{কারণ, $B_{i+1} - B_i > 0$}
\end{align*}
আসলেই তাই! (ইমপ্লিকেশন গুলো উল্টা অর্ডারে লিখতে হবে আরকি ফর্মাল প্রুফে...) তাহলে আমরা প্রুফ করে ফেললাম- এভাবে সোয়াপ করতে থাকলে আমরা কোন লস ছাড়াই অপ্টিমাল সলিউশন থেকে গ্রিডি সলিউশনে পৌছাতে পারবো (খেয়াল করো, শুধুমাত্র দুটো পাশাপাশি উপাদান সোয়াপ করে করেই কিন্তু একটি সিকুয়েন্সের যেকোনো পারমুটেশনে পৌছনো যায়)। অর্থাৎ, আমাদের গ্রিডি সলিউশনও একটি অপ্টিমাল সলিউশন!
\end{solution}

\section{মুল টেকনিক}

গ্রিডি অ্যালগরিদম বের করার পরে তা এক্সচেঞ্জ আর্গুমেন্ট দিয়ে প্রমাণ করার জন্য আমরা যা করি তাকে মূলত নিচের ৩টা স্টেপে ভাগ করা যায়-
\begin{enumerate}
	\item ধরলাম আমাদের গ্রিডি অ্যালগরিদম ব্যবহার করে আমরা একটা সলিউশন $G = \cbra{g_1, g_2, \ldots, g_n}$ পেয়েছি, আর $O = \cbra{o_1, o_2, \ldots, o_m}$ একটি অপ্টিমাল সলিউশন।  এখানে কিন্তু আমরা ধরে নিচ্ছি $G$ আর $O$ দুটোই সবরকমের শর্ত মেনেই বানানো হয়েছে।
	\item ধরে নাও $G \not= O$ আর  তাদের মধ্যে পার্থক্য করো, যেমন, ধর $G$ তে এমন একটি উপাদান পেলে যেটি $O$ তে নেই (অথবা, $O$ তে এমন একটি উপাদান পেলে যেটি $G$ তে নেই) অথবা এমন দুটি উপাদান আছে যারা $G$ তে যেই অর্ডারে আছে, $O$ তে তার বিপরীত অর্ডারে আছে।
	\item \textbf{এক্সচেঞ্জ।} যেমন, প্রথম কেইস এর জন্য $O$ থেকে একটি উপাদান বের করে আরেকটি উপাদান ঢুকালা, অথবা দ্বিতীয় কেইস এর জন্য অর্ডারটা সোয়াপ করে দিলে (বেশিরভাগ সময় খালি পাশাপাশি ২টা উপাদান নিয়েই কাজ করা হয়)। এখন কারণ দেখাও, এক্সচেঞ্জ করার পর তোমার নতুন সলিউশনটা আগেরটার তুলোনায় খারাপ না এবং এরপর দেখাবে তুমি যদি এইরকম এক্সচেঞ্জ করতে থাকো তাহলে একসময় $O$ কে $G$ এর সমান বানাতে পারবে। সুতরাং তোমার গ্রিডি সলিউশন যেকোনো অপ্টিমাল সলিউশনের (বা যেকোনো নন-অপ্টিমাল সলিউশনের) চাইতে ভাল বা সমান, যার মানে দাঁড়ালো তোমার সলিউশনও একটি অপ্টিমাল সলিউশন।
\end{enumerate}

অনেক ভারী ভারী আলোচনা হয়ে গেলো! আসলে প্রথমেই যে বলেছিলাম এক্সচেঞ্জ আর্গুমেন্ট দিয়ে প্রুফ করতে গিয়ে আমরা অনেকসময় গ্রিডি সলিউশনও দাঁড় করিয়ে ফেলতে পারি- এভাবে চিন্তা করলে আমরা কিছু কন্ডিশন পাই (যেমন পাশাপাশি ২টা উপাদানের মধ্যে কিরকম সম্পর্ক হতে পারে) এবং সেগুলো থেকে আমরা উপাদান গুলোর একটি অর্ডারিং পেতে পারি যেটা আমাদের কাজকে অনেক সহজ করে দেয়।
