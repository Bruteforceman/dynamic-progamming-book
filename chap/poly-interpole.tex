\chapter{পলিনমিয়াল ইন্টারপোলেশন}

\section{পলিনমিয়াল নিয়ে কিছু কথা}
তোমরা বহুপদী বা পলিনমিয়াল নিয়ে আগে হয়ত কাজ করেছ। সবচেয়ে বহুল প্রচলিত উদাহরণ হচ্ছে দ্বিঘাতী সমীকরণগুলো। যেমন ধর 
$$2x^2 + 5x - 15$$
এটি একটি দ্বিঘাতী পলিনমিয়াল (second degree)। আবার নিচের পলিনমিয়ালটি একটি ত্রিঘাতী পলিনমিয়াল (third degree) 
$$x^3 - 5x^2 + 2x + 3$$
সাধারণভাবে বলতে গেলে 
$$P(x) = \sum_{i=0}^n a_{i} x^{i} = a_n x^n + a_{n - 1} x^{n - 1} + \dots + a_1 x + a_{0}$$ 
একটি $n$ ঘাতী পলিনমিয়াল ($n$ th degree)। পাঠ্যবইয়ের ভাষায় বলতে গেলে একটি $n$ ঘাতী পলিনমিয়াল হল এমন একটি এক চলক বিশিষ্ট ফাংশন যার ঘাতগুলো অঋণাত্মক পুর্ণসংখ্যা এবং সর্বোচ্চ ঘাত $n$।

পলিনমিয়াল কী তা হয়ত সবাই বুঝতে পেরেছ। কিন্তু পলিনমিয়াল ইন্টারপোলেশন বলতে আসলে কি বুঝাচ্ছে। আমরা জানি পলিনমিয়ালগুলো বিশেষ ধরনের ফাংশন। ধর আমাদের একটা অজানা পলিনমিয়াল $P(x)$ বের করতে হবে। শুধু জানা আছে  $P(x)$ একটি তৃতীয় ঘাতী পলিনমিয়াল, এবং 
\begin{align*}
P(1) & = 21 \\
P(3) & = 4 \\
P(10) & = -5 \\
P(15) &= -8
\end{align*}
শুধু এটুকু জানলেই কী $P(x)$ কে বের করে ফেলা সম্ভব? উত্তর হচ্ছে হ্যাঁ। সাধারণভাবে বলা যায়, যদি আমরা পলিনমিয়ালের ডিগ্রি বা ঘাত সম্পর্কে জানি (ধর এই ডিগ্রি $n$), এবং $n + 1$ টি \textbf{ভিন্ন ভিন্ন} $x$ এর জন্য $P(x)$ এর মান জানি, তাহলে আমরা পলিনমিয়ালটিকে বের করে ফেলতে পারব (শুধু তাই নয়, সব শর্ত মেনে চলে এমন পলিনমিয়াল একটাই পাওয়া যাবে)। এই যে $n + 1$ টি $P(x)$ এর মান থেকে আমরা $n$ ডিগ্রি পলিনমিয়ালটিকে বের করে ফেললাম এই প্রসেসটাকেই বাল হয় পলিনমিয়াল ইন্টারপোলেশন। 

পরবর্তী সেকশনে যাওয়ার আগে পলিনমিয়ালের ডিগ্রির ব্যাপারে কিছু কথা বলে নেওয়া দরকার। যদিও এগুলো সবারই জানার কথা, তবুও পরবর্তীতে এটা অনেক জায়গায় কাজে লাগবে বলে আবার বলছি 
\begin{enumerate}
\item একটি $n$ ডিগ্রি পলিনমিয়ালের সাথে আরেকটা $m$ ডিগ্রি পলিনমিয়াল যোগ করলে যোগফলের ডিগ্রি হবে $\max{(n, m)}$। 
\item একটি $n$ ডিগ্রি পলিনমিয়ালের সাথে আরেকটা $m$ ডিগ্রি পলিনমিয়াল বিয়োগ করলে বিয়োগফলের সর্বোচ্চ ডিগ্রি হবে $\max{(n, m)}$। তবে এর চেয়ে কমও হতে পারে।
\item একটি $n$ ডিগ্রি পলিনমিয়ালের সাথে আরেকটা $m$ ডিগ্রি পলিনমিয়াল গুন করলে গুনফলের ডিগ্রি হবে $n + m$।
\end{enumerate}

\section{কীভাবে পলিনমিয়াল ইন্টারপোলেশন কাজ করে }
কীভাবে পলিনমিয়ালটাকে বের করতে পারব সেটা বুঝার জন্য শুরুতেই একটা সহজ উদাহরণ দেখা যাক। 
\begin{example}
এমন দ্বিঘাতী পলিনমিয়াল বের কর যেন 
\begin{align*}
P(1) & = -3 \\
P(4) & = 0 \\
P(5) & = 0 
\end{align*}
\end{example}
\begin{solution}
তোমরা এটা নিশ্চয় জানো যদি কোন বহুপদী বা পলিনমিয়াল $f(x)$ এর জন্য $f(a) = 0$ হয় তাহলে $(x - a)$ পলিনমিয়ালটির একটি উৎপাদক। আমরা এ জিনিশটিই এখানে ব্যবহার করব। প্রশ্ন অনুযায়ী 

\begin{align*}
P(4) = 0 \\
P(5) = 0
\end{align*} 

তার মানে $(x - 4)$ এবং $(x - 5)$ উভয়েই $P(x)$ এর উৎপাদক। তাই আমরা $P(x)$ কে এভাবে লিখতে পারি

$$ P(x) = (x - 4)(x - 5)Q$$

এখানে $Q$ কিন্তু একটি ধ্রুবক হবে। কারণ হল $(x - 4)$ এবং $(x - 5)$ এর গুণফল নিজেই একটি দ্বিঘাতী পলিনমিয়াল। তাই $Q$ এর ঘাত শূন্য হতে হবে (উভয় পাশে ডিগ্রি বা ঘাত সমান রাখার জন্য), অর্থাৎ $Q$ কে একটি ধ্রুবকই হতে হবে। উপরের সমীকরণে আমরা $x = 1$ বসালেই কিন্তু $Q$ এর মান বের করে ফেলতে পারব 

\begin{align*}
P(1) = (1 - 4)(1 - 5)Q & = -3 \\
       \Rightarrow   Q & = \frac{-3}{12}     
\end{align*} 
সুতরাং $P(x)$ এর মান হচ্ছে 
$$P(x) = \frac{-3}{12} (x - 4)(x - 5)$$
এখন একে বিস্তার (expand) করে দিলেই $P(x)$ এর সব সহগগুলো বের করে ফেলতে পারব। 
\end{solution}

এবার আরেকটু কঠিন উদাহরণ দেখা যাক 
\begin{example}
এমন দ্বিঘাতী পলিনমিয়াল বের কর যেন 
\begin{align*}
P(1) & = -1 \\
P(2) & = -5 \\
P(3) & = 3
\end{align*}
\end{example}

\begin{solution}
আগের উদাহরণটি আমাদের জন্য সহজ হয়ে গিয়েছিল কেন বল তো? কারণ ছিল একটি বাদে বাকি $P(x)$ গুলোর মান $0$ ছিল। তাই আমরা $P(x)$ এর সব উৎপাদক বের করে ফেলতে পেরেছিলাম। কিন্তু এখানে কোন $P(x) = 0$ নেই। তাহলে কী করা যায়? 

আমরা কিছুটা আগের উদাহরণের মতই চেষ্টা করব। ধরে নাও, শুধু $P(x) = -1$ বাকি $P(x)$ গুলোর মান $0$ (অর্থাৎ $P(2) = P(3) = 0$)। তাহলে আমরা আগের উদাহরণের মত একটি পলিনমিয়াম বের করতে পারব। এই পলিনমিয়ালের নাম দিলাম $P_1$। 

একইভাবে এবার ধর শুধু $P(2) = -5$, বাকি $P(x)$ গুলোর মান $0$ (অর্থাৎ $P(1) = P(3) = 0$)। এবারও আরেকটি পলিনমিয়াল $P_2$ বের হবে। 

শেষমেষ তৃতীয় পলিনমিয়াল $P_3$ বের করার জন্য $P(3) = 3$ এবং $P(1) = P(2) = 0$ ধরে নিয়ে সমধান করতে হবে। এভাবে আমরা তিনটি পলিনমিয়াল $P_1$, $P_2$, $P_3$ পেলাম। 

আমাদের কাজ কিন্তু প্রায় শেষ। এখন পলিনমিয়াল তিনটিকে যোগ করে দিলেই কাঙ্ক্ষিত পলিনমিয়ালটি পেয়ে যাব। অর্থাৎ 
$$P = P_1 + P_2 + P_3$$

এর কারণও খুব সহজ।
\begin{align*}
P(1) & = P_1(1) + P_2(1) + P_3(1) = (-1) + 0 + 0 = &-1 \\
P(2) & = P_1(2) + P_2(2) + P_3(2) = 0 + (-5) + 0 = &-5 \\
P(3) & = P_1(3) + P_2(3) + P_3(3) = 0 + 0 + (+3) = &3
\end{align*}

$P_1$, $P_2$, $P_3$ সবগুলোই $2$ ডিগ্রি পলিনমিয়াল হওয়ায় $P$ ও $2$ ডিগ্রি পলিনমিয়াল হবে। অর্থাৎ যেহেতু $P$ সব শর্ত সিদ্ধ করে করে, তাই এটিই নির্ণেয় উত্তর। 
\end{solution}

এখানে আমরা দ্বিঘাতী পলিনমিয়ালের জন্য ইন্টারপোলেশন করেছি। কিন্তু একই নিয়মে উপরের ঘাতের পলিনমিয়ালগুলোর জন্যও ইন্টারপোলেশন করা যাবে। 

\section{ল্যাগ্রাঞ্জ ইন্টারপোলেশন}
আমরা কিন্তু ল্যাগ্রাঞ্জ ইন্টারপলেশন ইতোমধ্যে শিখে ফেলেছি। আগের উদাহরণগুলোয় আমরা যেভাবে পলিনমিয়ালটা বের করেছি সেটার প্রচলিত নাম হচ্ছে ল্যাগ্রাঞ্জ ইন্টারপোলেশন। $n$ ডিগ্রি পলিনমিয়ালের জন্য আমাদের ইন্টারপোলেশন করতে হবে এভাবে: যদি
\begin{align*}
&P(x_0) = y_0 \\
&P(x_1) = y_1 \\
&P(x_2) = y_2 \\
&\vdots \\
&P(x_n) = y_n
\end{align*}
হয়, তাহলে $n$ ডিগ্রি পলিনমিয়াল $P(x)$ বের করার জন্য 
\begin{itemize}
\item প্রথমে প্রত্যেক $i$ এর জন্য $P(x_i) = y_i$ এবং $P(x_j) = 0$ (যেখানে $i \neq j$) ধরে নিয়ে একটি পলিনমিয়াল বের করতে হবে। অর্থাৎ আমরা এভাবে $n + 1$ টি $n$ ডিগ্রি পলিনমিয়াল পাব। আগের উদাহরণটির মত যদি সমাধান কর তাইলে দেখবে $i$ তম পলিনমিয়াল $P_i$ হবে
$$P_{i}(x) = y_{i} \times \prod_{\substack{j = 0 \\ i \neq j}}^n \frac{x - x_j}{x_i - x_j}}$$
\item এরপর প্রত্যেক পলিনমিয়ালকে বিস্তার করে দাও (এ কাজটি ফাস্ট ফুরিয়ার ট্রান্সফর্ম দিয়ে করা যায়; তবে আমাদের বইয়ের আলোচনার জন্য এটি দরকার নেই, $\mathcal{O}(n^2)$ কম্পেক্সিটিতে বিস্তার করাই যথেষ্ট)।
\item শেষ ধাপে আমাদের $n + 1$ টি পলিনমিয়াল যোগ করে দিতে হবে।  যোগফলই হবে আমাদের কাঙ্ক্ষিত পলিনমিয়াল। অর্থাৎ 
$$P(x) = \sum_{i=0}^{n} P_i(x)$$
\end{itemize}
এটাই ল্যাগ্রাঞ্জ ইন্টারপলেশনের অ্যালগরিদম। 

\section{ডাইনামিক প্রোগ্রামিং-এর সাথে সম্পর্ক}

