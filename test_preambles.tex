% This is the common preambles for doing any type of tests
% e.g. tinkering a tikz code (for example, see
% ./img/matrix-chain/perfect-bin-trees-with-5-leaves/test.tex)
% you can just start by doing:
% % This is the common preambles for doing any type of tests
% e.g. tinkering a tikz code (for example, see
% ./img/matrix-chain/perfect-bin-trees-with-5-leaves/test.tex)
% you can just start by doing:
% % This is the common preambles for doing any type of tests
% e.g. tinkering a tikz code (for example, see
% ./img/matrix-chain/perfect-bin-trees-with-5-leaves/test.tex)
% you can just start by doing:
% % This is the common preambles for doing any type of tests
% e.g. tinkering a tikz code (for example, see
% ./img/matrix-chain/perfect-bin-trees-with-5-leaves/test.tex)
% you can just start by doing:
% \input{path/to/test_preambles.tex}
% \begin{document}
% 	% your test codes
% \end{document}

\documentclass[twoside,12pt,a4paper]{book}

% \usepackage[a4paper,vmargin=30mm,hmargin=33mm,footskip=15mm,showframe]{geometry}
\usepackage[a4paper,vmargin=30mm,hmargin=33mm,footskip=15mm]{geometry}

\usepackage[utf8]{inputenc}

\usepackage{polyglossia}

\setmainlanguage[numerals=Bengali,changecounternumbering=true]{bengali}
\newfontfamily\bengalifont[Script=Bengali, Scale=0.9, NFSSFamily=bmfont, WordSpace=1.5, AutoFakeSlant, AutoFakeBold]{SolaimanLipi}
\newenvironment{latin}{\fontencoding{OT1}\ifx\f@family\btt@@name\fontfamily{lmtt}\else\fontfamily{lmr}\fi\selectfont}\relax
\RequirePackage[Latin, Bengali, Devanagari]{ucharclasses}\setTransitionsForLatin{\begin{latin}}{\end{latin}}
\newfontfamily\bengalifonttt{SolaimanLipi}

% \usepackage[english]{babel}
\usepackage{listings}
\usepackage[table]{xcolor}
\usepackage{tikz}
\usepackage{multicol}
\usepackage{hyperref}
\usepackage{array}
% \usepackage{microtype}

% \usepackage{fouriernc}
% \usepackage[T1]{fontenc}

\usepackage{graphicx}
\usepackage{framed}
\usepackage{amssymb}
\usepackage{amsmath}

% \usepackage{pifont}
\usepackage{ifthen}
\usepackage{makeidx}
\usepackage{enumitem}
\usepackage{tabularx} % https://pastebin.ubuntu.com/p/27JH8NrQGf/

\usepackage{titlesec}

% \usepackage{skak}
\usepackage[scaled=0.95]{inconsolata}

\pagestyle{plain}

% \usepackage{pdfpages}
% \usepackage{afterpage}
% \usepackage{bookstyle_exp}
% \usepackage{xcolor,colortbl}
% \usepackage{array}

% Hints config

\usepackage{answers}
\Newassociation{hint}{Hint}{hint_ostream}

% Code formatting

% \definecolor{keywords}{HTML}{44548A}
% \definecolor{strings}{HTML}{00999A}
% \definecolor{comments}{HTML}{990000}

\lstset{
  language=C++,
  frame=none,
  basicstyle=\ttfamily \small,
  commentstyle=\normalfont \itshape,
  showstringspaces=false,
  columns=flexible
}
\lstset{literate={^}{\caret}{1} {~}{\tilde}{1}}

\lstset{xleftmargin=5pt,xrightmargin=5pt}
\lstset{aboveskip=8pt,belowskip=6pt}

% \lstset{
%     commentstyle=\color{comments},
%     keywordstyle=\color{keywords},
%     stringstyle=\color{strings}
% }

\newcommand{\caret}{\textasciicircum} % usage \caret for ^
\newcommand{\ntilde}{\raise.17ex\hbox{$\scriptstyle\mathtt{\sim}$}} % failed to make \renewcommand*{\tilde} work :(, so gotta use this atm

% Macros

\newcommand{\cbra}[1]{\left \{ #1 \right \}} % c for curly bracket
\newcommand{\bbra}[1]{\left [ #1 \right ]} % b for boxed bracket
\newcommand{\pbra}[1]{\left ( #1 \right )} % p for parentheses
\newcommand{\floor}[1]{\left \lfloor #1 \right \rfloor}
\newcommand{\ceil}[1]{\left \lceil #1 \right \rceil}
\let\emptyset\varnothing % original emptyset is ugly

%%%

\title{\Huge ডাইনামিক প্রোগ্রামিং এ হাতেখড়ি}
\author{\Large তাসমীম রেজা \\ \Large মামনুন সিয়াম}
\date{Draft \today}

\newenvironment{solution}{\noindent \textit{সমাধান।}}{}
\newenvironment{diybox}{\textbf{নিজে করোঃ}}{}

\usepackage{amsthm}
% \usepackage{thmtools}

\theoremstyle{definition}
\newtheorem{theorem}{উপপাদ্য}[section]
\theoremstyle{definition}
\newtheorem*{theorem*}{উপপাদ্য}

\theoremstyle{definition}
\newtheorem{corollary}[theorem]{Corollary}
\theoremstyle{definition}
\newtheorem*{corollary*}{Corollary}

\theoremstyle{definition}
\newtheorem{lemma}[theorem]{লেমা}
\theoremstyle{definition}
\newtheorem*{lemma*}{লেমা}

\theoremstyle{definition}
\newtheorem{proposition}[theorem]{Proposition}
\theoremstyle{definition}
\newtheorem*{proposition*}{Proposition}

\theoremstyle{definition}
\newtheorem{claim}[theorem]{Claim}
\theoremstyle{definition}
\newtheorem*{claim*}{Claim}

\theoremstyle{definition}
\newtheorem{definition}[theorem]{সংজ্ঞা}
\theoremstyle{definition}
\newtheorem*{definition*}{সংজ্ঞা}

\theoremstyle{definition}
\newtheorem{problem}{প্রবলেম}[section]
\theoremstyle{definition}
\newtheorem*{problem*}{প্রবলেম}

\theoremstyle{definition}
\newtheorem{exercise}{অনুশীলনী}[section]
\theoremstyle{definition}
\newtheorem*{exercise*}{অনুশীলনী}

\theoremstyle{definition}
\newtheorem{example}{উদাহরণ}[section]
\theoremstyle{definition}
\newtheorem*{example*}{উদাহরণ}

\newtheorem*{note*}{Note}

% \begin{document}
% 	% your test codes
% \end{document}

\documentclass[twoside,12pt,a4paper]{book}

% \usepackage[a4paper,vmargin=30mm,hmargin=33mm,footskip=15mm,showframe]{geometry}
\usepackage[a4paper,vmargin=30mm,hmargin=33mm,footskip=15mm]{geometry}

\usepackage[utf8]{inputenc}

\usepackage{polyglossia}

\setmainlanguage[numerals=Bengali,changecounternumbering=true]{bengali}
\newfontfamily\bengalifont[Script=Bengali, Scale=0.9, NFSSFamily=bmfont, WordSpace=1.5, AutoFakeSlant, AutoFakeBold]{SolaimanLipi}
\newenvironment{latin}{\fontencoding{OT1}\ifx\f@family\btt@@name\fontfamily{lmtt}\else\fontfamily{lmr}\fi\selectfont}\relax
\RequirePackage[Latin, Bengali, Devanagari]{ucharclasses}\setTransitionsForLatin{\begin{latin}}{\end{latin}}
\newfontfamily\bengalifonttt{SolaimanLipi}

% \usepackage[english]{babel}
\usepackage{listings}
\usepackage[table]{xcolor}
\usepackage{tikz}
\usepackage{multicol}
\usepackage{hyperref}
\usepackage{array}
% \usepackage{microtype}

% \usepackage{fouriernc}
% \usepackage[T1]{fontenc}

\usepackage{graphicx}
\usepackage{framed}
\usepackage{amssymb}
\usepackage{amsmath}

% \usepackage{pifont}
\usepackage{ifthen}
\usepackage{makeidx}
\usepackage{enumitem}
\usepackage{tabularx} % https://pastebin.ubuntu.com/p/27JH8NrQGf/

\usepackage{titlesec}

% \usepackage{skak}
\usepackage[scaled=0.95]{inconsolata}

\pagestyle{plain}

% \usepackage{pdfpages}
% \usepackage{afterpage}
% \usepackage{bookstyle_exp}
% \usepackage{xcolor,colortbl}
% \usepackage{array}

% Hints config

\usepackage{answers}
\Newassociation{hint}{Hint}{hint_ostream}

% Code formatting

% \definecolor{keywords}{HTML}{44548A}
% \definecolor{strings}{HTML}{00999A}
% \definecolor{comments}{HTML}{990000}

\lstset{
  language=C++,
  frame=none,
  basicstyle=\ttfamily \small,
  commentstyle=\normalfont \itshape,
  showstringspaces=false,
  columns=flexible
}
\lstset{literate={^}{\caret}{1} {~}{\tilde}{1}}

\lstset{xleftmargin=5pt,xrightmargin=5pt}
\lstset{aboveskip=8pt,belowskip=6pt}

% \lstset{
%     commentstyle=\color{comments},
%     keywordstyle=\color{keywords},
%     stringstyle=\color{strings}
% }

\newcommand{\caret}{\textasciicircum} % usage \caret for ^
\newcommand{\ntilde}{\raise.17ex\hbox{$\scriptstyle\mathtt{\sim}$}} % failed to make \renewcommand*{\tilde} work :(, so gotta use this atm

% Macros

\newcommand{\cbra}[1]{\left \{ #1 \right \}} % c for curly bracket
\newcommand{\bbra}[1]{\left [ #1 \right ]} % b for boxed bracket
\newcommand{\pbra}[1]{\left ( #1 \right )} % p for parentheses
\newcommand{\floor}[1]{\left \lfloor #1 \right \rfloor}
\newcommand{\ceil}[1]{\left \lceil #1 \right \rceil}
\let\emptyset\varnothing % original emptyset is ugly

%%%

\title{\Huge ডাইনামিক প্রোগ্রামিং এ হাতেখড়ি}
\author{\Large তাসমীম রেজা \\ \Large মামনুন সিয়াম}
\date{Draft \today}

\newenvironment{solution}{\noindent \textit{সমাধান।}}{}
\newenvironment{diybox}{\textbf{নিজে করোঃ}}{}

\usepackage{amsthm}
% \usepackage{thmtools}

\theoremstyle{definition}
\newtheorem{theorem}{উপপাদ্য}[section]
\theoremstyle{definition}
\newtheorem*{theorem*}{উপপাদ্য}

\theoremstyle{definition}
\newtheorem{corollary}[theorem]{Corollary}
\theoremstyle{definition}
\newtheorem*{corollary*}{Corollary}

\theoremstyle{definition}
\newtheorem{lemma}[theorem]{লেমা}
\theoremstyle{definition}
\newtheorem*{lemma*}{লেমা}

\theoremstyle{definition}
\newtheorem{proposition}[theorem]{Proposition}
\theoremstyle{definition}
\newtheorem*{proposition*}{Proposition}

\theoremstyle{definition}
\newtheorem{claim}[theorem]{Claim}
\theoremstyle{definition}
\newtheorem*{claim*}{Claim}

\theoremstyle{definition}
\newtheorem{definition}[theorem]{সংজ্ঞা}
\theoremstyle{definition}
\newtheorem*{definition*}{সংজ্ঞা}

\theoremstyle{definition}
\newtheorem{problem}{প্রবলেম}[section]
\theoremstyle{definition}
\newtheorem*{problem*}{প্রবলেম}

\theoremstyle{definition}
\newtheorem{exercise}{অনুশীলনী}[section]
\theoremstyle{definition}
\newtheorem*{exercise*}{অনুশীলনী}

\theoremstyle{definition}
\newtheorem{example}{উদাহরণ}[section]
\theoremstyle{definition}
\newtheorem*{example*}{উদাহরণ}

\newtheorem*{note*}{Note}

% \begin{document}
% 	% your test codes
% \end{document}

\documentclass[twoside,12pt,a4paper]{book}

% \usepackage[a4paper,vmargin=30mm,hmargin=33mm,footskip=15mm,showframe]{geometry}
\usepackage[a4paper,vmargin=30mm,hmargin=33mm,footskip=15mm]{geometry}

\usepackage[utf8]{inputenc}

\usepackage{polyglossia}

\setmainlanguage[numerals=Bengali,changecounternumbering=true]{bengali}
\newfontfamily\bengalifont[Script=Bengali, Scale=0.9, NFSSFamily=bmfont, WordSpace=1.5, AutoFakeSlant, AutoFakeBold]{SolaimanLipi}
\newenvironment{latin}{\fontencoding{OT1}\ifx\f@family\btt@@name\fontfamily{lmtt}\else\fontfamily{lmr}\fi\selectfont}\relax
\RequirePackage[Latin, Bengali, Devanagari]{ucharclasses}\setTransitionsForLatin{\begin{latin}}{\end{latin}}
\newfontfamily\bengalifonttt{SolaimanLipi}

% \usepackage[english]{babel}
\usepackage{listings}
\usepackage[table]{xcolor}
\usepackage{tikz}
\usepackage{multicol}
\usepackage{hyperref}
\usepackage{array}
% \usepackage{microtype}

% \usepackage{fouriernc}
% \usepackage[T1]{fontenc}

\usepackage{graphicx}
\usepackage{framed}
\usepackage{amssymb}
\usepackage{amsmath}

% \usepackage{pifont}
\usepackage{ifthen}
\usepackage{makeidx}
\usepackage{enumitem}
\usepackage{tabularx} % https://pastebin.ubuntu.com/p/27JH8NrQGf/

\usepackage{titlesec}

% \usepackage{skak}
\usepackage[scaled=0.95]{inconsolata}

\pagestyle{plain}

% \usepackage{pdfpages}
% \usepackage{afterpage}
% \usepackage{bookstyle_exp}
% \usepackage{xcolor,colortbl}
% \usepackage{array}

% Hints config

\usepackage{answers}
\Newassociation{hint}{Hint}{hint_ostream}

% Code formatting

% \definecolor{keywords}{HTML}{44548A}
% \definecolor{strings}{HTML}{00999A}
% \definecolor{comments}{HTML}{990000}

\lstset{
  language=C++,
  frame=none,
  basicstyle=\ttfamily \small,
  commentstyle=\normalfont \itshape,
  showstringspaces=false,
  columns=flexible
}
\lstset{literate={^}{\caret}{1} {~}{\tilde}{1}}

\lstset{xleftmargin=5pt,xrightmargin=5pt}
\lstset{aboveskip=8pt,belowskip=6pt}

% \lstset{
%     commentstyle=\color{comments},
%     keywordstyle=\color{keywords},
%     stringstyle=\color{strings}
% }

\newcommand{\caret}{\textasciicircum} % usage \caret for ^
\newcommand{\ntilde}{\raise.17ex\hbox{$\scriptstyle\mathtt{\sim}$}} % failed to make \renewcommand*{\tilde} work :(, so gotta use this atm

% Macros

\newcommand{\cbra}[1]{\left \{ #1 \right \}} % c for curly bracket
\newcommand{\bbra}[1]{\left [ #1 \right ]} % b for boxed bracket
\newcommand{\pbra}[1]{\left ( #1 \right )} % p for parentheses
\newcommand{\floor}[1]{\left \lfloor #1 \right \rfloor}
\newcommand{\ceil}[1]{\left \lceil #1 \right \rceil}
\let\emptyset\varnothing % original emptyset is ugly

%%%

\title{\Huge ডাইনামিক প্রোগ্রামিং এ হাতেখড়ি}
\author{\Large তাসমীম রেজা \\ \Large মামনুন সিয়াম}
\date{Draft \today}

\newenvironment{solution}{\noindent \textit{সমাধান।}}{}
\newenvironment{diybox}{\textbf{নিজে করোঃ}}{}

\usepackage{amsthm}
% \usepackage{thmtools}

\theoremstyle{definition}
\newtheorem{theorem}{উপপাদ্য}[section]
\theoremstyle{definition}
\newtheorem*{theorem*}{উপপাদ্য}

\theoremstyle{definition}
\newtheorem{corollary}[theorem]{Corollary}
\theoremstyle{definition}
\newtheorem*{corollary*}{Corollary}

\theoremstyle{definition}
\newtheorem{lemma}[theorem]{লেমা}
\theoremstyle{definition}
\newtheorem*{lemma*}{লেমা}

\theoremstyle{definition}
\newtheorem{proposition}[theorem]{Proposition}
\theoremstyle{definition}
\newtheorem*{proposition*}{Proposition}

\theoremstyle{definition}
\newtheorem{claim}[theorem]{Claim}
\theoremstyle{definition}
\newtheorem*{claim*}{Claim}

\theoremstyle{definition}
\newtheorem{definition}[theorem]{সংজ্ঞা}
\theoremstyle{definition}
\newtheorem*{definition*}{সংজ্ঞা}

\theoremstyle{definition}
\newtheorem{problem}{প্রবলেম}[section]
\theoremstyle{definition}
\newtheorem*{problem*}{প্রবলেম}

\theoremstyle{definition}
\newtheorem{exercise}{অনুশীলনী}[section]
\theoremstyle{definition}
\newtheorem*{exercise*}{অনুশীলনী}

\theoremstyle{definition}
\newtheorem{example}{উদাহরণ}[section]
\theoremstyle{definition}
\newtheorem*{example*}{উদাহরণ}

\newtheorem*{note*}{Note}

% \begin{document}
% 	% your test codes
% \end{document}

\documentclass[twoside,12pt,a4paper]{book}

\usepackage[a4paper,vmargin=30mm,hmargin=33mm,footskip=15mm,showframe]{geometry}

\usepackage[utf8]{inputenc}

\usepackage{polyglossia}

\setmainlanguage[numerals=Bengali,changecounternumbering=true]{bengali}
\newfontfamily\bengalifont[Script=Bengali, Scale=0.9, NFSSFamily=bmfont, WordSpace=1.5, AutoFakeSlant, AutoFakeBold]{SolaimanLipi}
\newenvironment{latin}{\fontencoding{OT1}\ifx\f@family\btt@@name\fontfamily{lmtt}\else\fontfamily{lmr}\fi\selectfont}\relax
\RequirePackage[Latin, Bengali, Devanagari]{ucharclasses}\setTransitionsForLatin{\begin{latin}}{\end{latin}}
\newfontfamily\bengalifonttt{SolaimanLipi}

% \usepackage[english]{babel}
\usepackage{listings}
\usepackage[table]{xcolor}
\usepackage{tikz}
\usepackage{multicol}
\usepackage{hyperref}
\usepackage{array}
% \usepackage{microtype}

% \usepackage{fouriernc}
% \usepackage[T1]{fontenc}

\usepackage{graphicx}
\usepackage{framed}
\usepackage{amssymb}
\usepackage{amsmath}

% \usepackage{pifont}
\usepackage{ifthen}
\usepackage{makeidx}
\usepackage{enumitem}
\usepackage{tabularx} % https://pastebin.ubuntu.com/p/27JH8NrQGf/

\usepackage{titlesec}

% \usepackage{skak}
\usepackage[scaled=0.95]{inconsolata}

\pagestyle{plain}

% \usepackage{pdfpages}
% \usepackage{afterpage}
% \usepackage{bookstyle_exp}
% \usepackage{xcolor,colortbl}
% \usepackage{array}

% Hints config

\usepackage{answers}
\Newassociation{hint}{Hint}{hint_ostream}

% Code formatting

% \definecolor{keywords}{HTML}{44548A}
% \definecolor{strings}{HTML}{00999A}
% \definecolor{comments}{HTML}{990000}

\lstset{
  language=C++,
  frame=none,
  basicstyle=\ttfamily \small,
  commentstyle=\normalfont \itshape,
  showstringspaces=false,
  columns=flexible
}
\lstset{literate={^}{\caret}{1} {~}{\tilde}{1}}

\lstset{xleftmargin=5pt,xrightmargin=5pt}
\lstset{aboveskip=8pt,belowskip=6pt}

% \lstset{
%     commentstyle=\color{comments},
%     keywordstyle=\color{keywords},
%     stringstyle=\color{strings}
% }

\newcommand{\caret}{\textasciicircum} % usage \caret for ^
\newcommand{\ntilde}{\raise.17ex\hbox{$\scriptstyle\mathtt{\sim}$}} % failed to make \renewcommand*{\tilde} work :(, so gotta use this atm

% Macros

\newcommand{\cbra}[1]{\left \{ #1 \right \}} % c for curly bracket
\newcommand{\bbra}[1]{\left [ #1 \right ]} % b for boxed bracket
\newcommand{\pbra}[1]{\left ( #1 \right )} % p for parentheses
\newcommand{\floor}[1]{\left \lfloor #1 \right \rfloor}
\newcommand{\ceil}[1]{\left \lceil #1 \right \rceil}
\let\emptyset\varnothing % original emptyset is ugly

%%%

\title{\Huge ডাইনামিক প্রোগ্রামিং এ হাতেখড়ি}
\author{\Large তাসমীম রেজা \\ \Large মামনুন সিয়াম}
\date{Draft \today}

\newenvironment{solution}{\noindent \textit{সমাধান।}}{}
\newenvironment{diybox}{\textbf{নিজে করোঃ}}{}

\usepackage{amsthm}
% \usepackage{thmtools}

\theoremstyle{definition}
\newtheorem{theorem}{উপপাদ্য}[section]
\theoremstyle{definition}
\newtheorem*{theorem*}{উপপাদ্য}

\theoremstyle{definition}
\newtheorem{corollary}[theorem]{Corollary}
\theoremstyle{definition}
\newtheorem*{corollary*}{Corollary}

\theoremstyle{definition}
\newtheorem{lemma}[theorem]{লেমা}
\theoremstyle{definition}
\newtheorem*{lemma*}{লেমা}

\theoremstyle{definition}
\newtheorem{proposition}[theorem]{Proposition}
\theoremstyle{definition}
\newtheorem*{proposition*}{Proposition}

\theoremstyle{definition}
\newtheorem{claim}[theorem]{Claim}
\theoremstyle{definition}
\newtheorem*{claim*}{Claim}

\theoremstyle{definition}
\newtheorem{definition}[theorem]{সংজ্ঞা}
\theoremstyle{definition}
\newtheorem*{definition*}{সংজ্ঞা}

\theoremstyle{definition}
\newtheorem{problem}{প্রবলেম}[section]
\theoremstyle{definition}
\newtheorem*{problem*}{প্রবলেম}

\theoremstyle{definition}
\newtheorem{exercise}{অনুশীলনী}[section]
\theoremstyle{definition}
\newtheorem*{exercise*}{অনুশীলনী}

\theoremstyle{definition}
\newtheorem{example}{উদাহরণ}[section]
\theoremstyle{definition}
\newtheorem*{example*}{উদাহরণ}

\newtheorem*{note*}{Note}
